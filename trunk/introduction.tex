\newpage 
\section{Introduction}

It is well established from astrophysical observations that most of the matter in the Universe is dark matter (DM)~\cite{FNAL_Review}.
However, its particle nature remains unknown and cannot be accommodated within the standard model (SM). One of the leading hypothesis about the dark matter foresees stable, electrical neutral, massive particles which interact with baryons at least via the gravitational force. 

There is considerable interest in the idea that the DM particle interacts with SM states via the exchange of one or more new mediators~\cite{Frandsen2012,Alves2014,Garny2014}, which can for example carry spin 1 (e.g. a new Z' gauge boson) or spin 0 (e.g. an additional Higgs boson). The presence of such new mediators can lead to observable signals in a wide range of DM searches, in particular direct~\cite{PhysRevLett.117.121303,PhysRevLett.118.021303,PhysRevD.93.052014} and indirect detection experiments~\cite{PhysRevLett.115.231301,PhysRevLett.117.091103} and searches for missing transverse momentum at the Large Hadron Collider (LHC)~\cite{Buchmueller2015}. These mediators could also be responsible for establishing thermal equilibrium between the visible and the dark sector in the early Universe and provide the annihilation and creation processes that set the DM relic abundance via thermal freeze-out~\cite{Chala2015}. 

A particular interesting case where the DM mass is generated via Higgs mechanism in the dark sector, the resulted dark Higgs boson could be lighter then DM and later decay into SM state. Thus relaxing the already constrained phase space~\cite{Duerr2016}. A promising way to probe these models opens up if there is another mechanism to produce dark sector states at the LHC (for example via an additional Z' mediator), because any such state can radiate off a dark Higgs boson. Since the couplings within the dark sector are typically large in order to reproduce the observed relic abundance, the probability of dark-Higgs strahlung can be large despite the very small couplings of the dark Higgs boson to the SM. If the dark Higgs boson is the lightest state in the dark sector it further decays into SM particles. The emission of a visibly decaying dark Higgs boson then indicates the production of DM.

%If such a DM particle also interacts non-gravitationally with standard model particles, then DM could be produced at LHC, CERN. The classical way to search for DM particles at colliders is through their recoil against a SM particle X (X = g, q, $\gamma$, W, Z, or h) that is produced in association with the DM particles. This associated production of DM and SM particles is often referred to as mono-X production. The SM particle X can be emitted directly from a quark or gluon as initial-state radiation, or through a new interaction between DM and SM particles, or as final-state radiation. The Higgs boson  radiation from an initial-state quark or gluon is suppressed through Yukawa or loop processes, respectively.

%A scheme in which the Higgs boson is part of the interaction producing the DM particles gives rise to mono-h searches. The mono-h signal will uniquely enhance the sensitivity to the structure of couplings between the SM particles and DM. The recent discovery of a Higgs boson with mass of about 125\,GeV by the ATLAS and CMS experiments~\cite{HiggsObs_ATLAS, HiggsObs_CMS, HiggsObs_CMS_Long} provides an additional handle and motivation to probe the dark sector. This analysis uses the Higgs boson to search for the dark matter at CMS. If DM is associated with the electroweak symmetry breaking scale, Higgs boson-related signatures are a natural place to search for it.

We present an analysis searching for dark matter production, Results are presented for the full dataset of approximately 36\fbinv collected by CMS at a center of mass energy of 13\TeV~during the year 2016. 

\newpage
\section{Analysis strategy in general terms}

For identifying a possible dark Higgs boson production, two final state processes Mono-dark-Higgs decay into two bottom quarks and Mono-Z' decays into two light quarks will be studied. The discriminating variable employed is the missing transverse energy (\ptmiss) which would yield a harder \ptmiss spectrum from SM background for the case of Mono-Z'; and the invariant mass of two leading jets, which is expected to have a resonance structure in invariant jet mass spectrum distinguishing from SM backgrounds for the case of mono-dark-Higgs. In order to derive a model to constrain their uncertainties, a sophisticated maximum-likelihood fit is performed simultaneously in a region characterized by signal-enriching cuts and in control regions obtained by inverting requirements on lepton multiplicit. The Mono-dark-Higgs signal region is enriched with the requirement of b tagging, on the other hand, Mono-Z'-enriched signal region is obtained by inverting requirements of b tagging. The main backgrounds \ttbar, Z+jets, and W+jets are thus derived in a data-driven way by having bin-by-bin transfer factors which are allowed to modify the yields for a specific background in a specific bin of the \ptmiss distribution and the yields for this background in the recoil distribution of dedicated control regions in a correlated way. These transfer factors are afflicted with statistical and systematic uncertainties. The Higgs boson candidate is identified by clustering jets with a radius $R=1.5$ with the Cambridge-Aachen algorithm, and by imposing requirements on the number of prongs and the b quark content inside this ```fat jet''. Appropriate scale factors to account for potentially mismodelling these variables in simulation are derived either in standalone measurements in orthogonal samples or are obtained in-situ during the signal extraction by employing the aforementioned control regions with additional parameters tying them together and thereby steering the relative normalizations. 


