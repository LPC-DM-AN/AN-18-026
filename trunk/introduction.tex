\newpage 
\section{Introduction}

It is well established from astrophysical observations that most of the matter in the Universe is dark matter (DM)~\cite{FNAL_Review}.
However, its particle nature remains unknown and cannot be accommodated within the standard model (SM). One of the leading hypothesis about the dark matter foresees stable, electrical neutral, massive particles which interact with baryons at least via the gravitational force. If such a DM particle
also interacts non-gravitationally with standard model particles, then DM could be produced at LHC, CERN. The classical way to search for DM particles at colliders
is through their recoil against a SM particle X (X = g, q, $\gamma$, W, Z, or h) that is produced in association with the DM particles.
This associated production of DM and SM particles is often referred to as mono-X production. The SM particle X can be emitted directly from a quark or
gluon as initial-state radiation, or through a new interaction between DM and SM particles, or as final-state radiation. The Higgs boson
 radiation from an initial-state quark or gluon is suppressed through Yukawa or loop processes, respectively.

A scheme in which the Higgs boson is part of the interaction producing the DM particles gives rise to mono-h searches. The mono-h signal will uniquely enhance
the sensitivity to the structure of couplings between the SM particles and DM. The recent discovery of a Higgs boson with mass of
about 125\,GeV by the ATLAS and CMS experiments~\cite{HiggsObs_ATLAS, HiggsObs_CMS, HiggsObs_CMS_Long} provides an additional handle and motivation to probe the dark sector. This analysis uses the Higgs boson to search for the dark matter at CMS. If DM is associated with the electroweak symmetry breaking scale, Higgs boson-related signatures are a natural
place to search for it.

We present an analysis searching for dark matter production in association with a SM Higgs boson decaying into a pair of bottom quark. The H$\rightarrow \text{b} \bar{\text{b}}$ decay channel is important to study as it gives the highest signal yields of all possible Higgs boson decay channels due to the large branching ratio of the Higgs boson to b quarks. Results are presented for the full dataset of approximately 36\fbinv collected by CMS at a center of mass energy of 13\TeV~during the year 2016. 

\newpage
\section{Analysis strategy in general terms}

For identifying a possible mono-H signal, the discriminating variable employed is the missing transverse energy (\ptmiss), which is expected to have a harder spectrum for a mono-H process than for SM backgrounds. In order to derive a model for the latter and to constrain their uncertainties, a sophisticated maximum-likelihood fit is performed simultaneously in a region characterized by signal-enriching cuts and in control regions obtained by inverting requirements on lepton multiplicity and b tagging. The main backgrounds \ttbar, Z+jets, and W+jets are thus derived in a data-driven way by having bin-by-bin transfer factors which are allowed to modify the yields for a specific background in a specific bin of the \ptmiss distribution and the yields for this background in the recoil distribution of dedicated control regions in a correlated way. These transfer factors are afflicted with statistical and systematic uncertainties. The Higgs boson candidate is identified by clustering jets with a radius $R=1.5$ with the Cambridge-Aachen algorithm, and by imposing requirements on the number of prongs and the b quark content inside this ```fat jet''. Appropriate scale factors to account for potentially mismodelling these variables in simulation are derived either in standalone measurements in orthogonal samples or are obtained in-situ during the signal extraction by employing the aforementioned control regions with additional parameters tying them together and thereby steering the relative normalizations. 


